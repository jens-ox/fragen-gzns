\section*{Sinnesorgane}

\paragraph{Frage 1 (single choice)}

Welche Aussage zur Tränenflüssigkeit trifft zu?

\begin{benumerate}
  \item Der Sehnerv (\emph{nervus opticus}) steuert die Tränendrüsen.
  \item Direkte Reizung von Schmerzrezeptoren in der Hornhaut (\emph{cornea}) führt reflektorisch zu einer verminderten Tränensekretion.
  \item Über die beiden Tränekanäle an jedem Auge gelangt die Tränenflüssigkeit in den Liquorraum (Hirnwassersystem).
  \item Die Tränenflüssigkeit hat den gleichen pH-Wert wie Magensaft.
  \bolditem Die Tränenflüssigkeit fließt von der Tränendrüse über die \emph{canaliculi lacrimales superior} und \emph{inferior} (oberer und unterer Tränenkanal) in die Nasenhöhle ab.
\end{benumerate}

\paragraph{Frage 2 (single choice)}

Welche Komponente dient als molekularer Photorezeptor in den Stäbchenzellen der Retina?

1. Adenosin \enskip 2. Gabapentin \enskip 3. Glycin \enskip 4. Transducin \enskip \textbf{5.} Rhodopsin

\paragraph{Frage 3 (single choice)}

Welche Aussage zum Geschmackssinn trifft zu?
\begin{benumerate}
  \item Die menschliche Zunge unterscheidet 4 Geschmacksrichtungen: süß, sauer, salzig und scharf
  \item Der Geruchssinn und der Geschmackssinn sind anatomisch und topographisch im selben Hirnareal lokalisiert
  \bolditem Dem Trigeminusnerv sind keine spezifischen Sinneszellen zuzuordnen, sondern freie Nervenendungen, die auch für die Sensibilität im Gesicht verantwortlich sind
  \item Bitterstoffe wie Chinin werden nur über \emph{nervus olfactorius} wahrgenommen
  \item Guter Geschmack ist immer sofort zu erkennen
\end{benumerate}

\paragraph{Frage 4 (single choice)}

Welche Aussage zur Schädigung des Riechvermögens eines Menschen trifft am wahrscheinlichsten zu?
\begin{benumerate}
  \item Sinneszellen des Riechepithels werden nach einer durchschnittlichen Lebensdauer von 1-2 Tagen durch Zellteilung erneuert.
  \item Die Axone von Sinneszellen des Riechepithels projizieren überwiegend ohne Umschaltung zum Kleinhirn.
  \item Die subjektive Empfindung des Geschmacks von Speisen bleibt unverändert.
  \item Der Hustenreflex ist nicht mehr auslösbar.
  \bolditem Der Trigeminusreizstoff Ammoniak bleibt bei Abriss der Riechnerven (\emph{fila olfactoria}) typischerweise wahrnehmbar.
\end{benumerate}