\section*{Sinnesorgane}

\paragraph{Frage 1 (single choice)}

Welche Aussage zur Tränenflüssigkeit trifft zu?

\begin{benumerate}
  \item Der Sehnerv (\emph{nervus opticus}) steuert die Tränendrüsen.
  \item Direkte Reizung von Schmerzrezeptoren in der Hornhaut (\emph{cornea}) führt reflektorisch zu einer verminderten Tränensekretion.
  \item Über die beiden Tränekanäle an jedem Auge gelangt die Tränenflüssigkeit in den Liquorraum (Hirnwassersystem).
  \item Die Tränenflüssigkeit hat den gleichen pH-Wert wie Magensaft.
  \bolditem Die Tränenflüssigkeit fließt von der Tränendrüse über die \emph{canaliculi lacrimales superior} und \emph{inferior} (oberer und unterer Tränenkanal) in die Nasenhöhle ab.
\end{benumerate}