\section*{Neurodiagnostik}

\paragraph{Frage 1 (single choice)}

Für die Beurteilung der Wirkung ionisierender Strahlung auf den menschlichen Körper ist  die Äquivalentdosis eine wichtige Größe. In welcher Einheit wird die Äquivalentdosis angegeben?
\begin{benumerate}
  \item Bq (Becquerel)
  \item C/kg (Coulomb pro Kilogramm)
  \bolditem Sv (Sievert)
  \item Gy/Rd (Gray pro Rad)
\end{benumerate}

\paragraph{Frage 2 (single choice)}

In einer Röntgenanlage zur medizinischen Diagnostik mit einer typischen Röntgenröhre wird durch einen Generator aus der Netzspannung 220 Volt eine Hochspannung (in der Größenordnung von 100 kV) erzeugt. Für welchen der folgenden Vorgänge wird dabei aus prinzipiellen Gründen umbedingt Hochspannung benötigt?

\begin{benumerate}
  \item Abschirmung der Röntgenröhre
  \bolditem Beschleunigung der freien Elektronen in der Röntgenröhre
  \item Einstellung der Blenden
  \item hochtourige Drehung des Anodentellers in der Röntgenröhre
  \item Kühlung der Röntgenröhre
\end{benumerate}

\paragraph{Frage 3 (single choice)}

Welche Aussage zur Elektroenzephalographie (EEG) trifft zu?

\begin{benumerate}
  \item Es entspricht im Wesentlichen summierten synchronen Aktionspotentialen von oberflächlichen alpha-Motoneuronen.
  \item Bei Erwachsenen im inaktiven Wachzustand mit geschlossenen Augen lässt sich keinerlei elektrische Aktivität nachweisen (Nulllinie).
  \item Bei wachen, aufmerksamen Erwachsenen mit offenen Augen finden sich überwiegend Subdelta-Wellen.
  \bolditem Bei einem epileptischen Anfall können typische spikes und sharp waves auftreten.
  \item Aufgrund geringer Synchronisierung der neuronalen Aktivität ist die dominierende Frequenz der Wellen im Wachzustand bei Kleinkindern um ein vielfaches höher als bei Erwachsenen.
\end{benumerate}

\paragraph{Frage 4 (single choice)}

Welche Aussage zur Untersuchung des Gehirns und Rückenmarks mittels Magnetresonanztomographie (MRT) ist zutreffend?
\begin{benumerate}
  \item MRT ist eine typische nuklearmedizinische Untersuchung.
  \item MRT wird speziell bei Patienten mit Herzschrittmachern eingesetzt.
  \item Die unterschiedlichen Relaxationszeiten spielen keine Rolle für den Bildkontrast.
  \item Die angewandte ionisierende Strahlung ist aufgrund der Abschirmung der Spulen unschädlich für den untersuchten Patienten.
  \bolditem Der Spin der Wasserstoffatome erzeugt durch ihren eigenen Drehimpuls ein kleines Magnetfeld.
\end{benumerate}

\paragraph{Frage 5 (single choice)}

Es soll die Nervenleitgeschwindigkeit ermittelt werden. Daher wird bei einem Patienten der \emph{N. ulnaris} am Oberarm und am Handgelenk gereizt und jeweils das Summenaktionspotential abgeleitet. Bei Reizung am Oberarm beginnt das Summenaktionspotential nach \( 10.5 \) ms und bei Reizung am Handgelenk nach \( 2.1 \) ms. Die beiden Reizorte sind exakt \( 42 \) cm voneinander entfernt. Die motorische Nervenleitgeschwindigkeit ist die mittlere Erregungsleitgeschwindigkeit zwischen den beiden Reizorten. Wie groß ist sie bei dem Patienten?
\begin{benumerate}
  \item 20 m/s
  \item 30 m/s
  \item 40 m/s
  \bolditem 50 m/s
  \item 60 m/s
\end{benumerate}

\paragraph{Frage 6 (single choice)}

Welche der folgenden Aussagen zur natürlich vorkommenden Strahlung ist zutreffend?
\begin{benumerate}
  \item Terrestrische Strahlung ist auf der Erdoberfläche eine konstante Messgröße.
  \item Kosmische Strahlung entsteht in der inneren Atmosphäre und ist bei Langstreckenflügen geringer als am Boden.
  \item Die natürliche Strahlenbelastung ist eine Summation aus der inneren Strahlung und medizinischer Strahlenexposition und steigt in Deutschland kontinuierlich.
  \bolditem Die aufgenommene Menge an Radon und deren radioaktiven Zerfallsprodukten in der Atemluft, im Trinkwasser und in Nahrungsmitteln, verursachen den Hauptanteil der natürlichen Strahlungsexposition in Deutschland.
  \item Radionuklide wie Thorium und Polonium sind die beiden radioaktiven Stoffe die den größten Anteil der natürlichen Strahlenexposition in Europa verursachen.
\end{benumerate}

\paragraph{Frage 7 (one combination)}

Welche Verfahren sind in der medizinischen Diagnostik gebräuchliche Röntgenverfahren, die mit ionisierenden Strahlen arbeiten?

\begin{minipage}{.55\linewidth}
  \begin{benumerate}
    \item Coloskopie
    \item Mammographie
    \item Digitale Subtraktionsangiographie (DSA)
    \item Transcranielle Dopplersonographie (TCD)
    \item Intraoperative Durchleuchtung der Lendenwirbelsäule mittels Bildwandler
  \end{benumerate}
\end{minipage}
\hfill
\begin{minipage}{.4\linewidth}
  \begin{checklist}
    \item 1 und 3 ist richtig
    \item 2 und 4 ist richtig
    \item[\checkedbox] 2, 3 und 5 ist richtig
    \item nur 5 ist richtig
    \item alle Aussagen sind richtig
  \end{checklist}
\end{minipage}