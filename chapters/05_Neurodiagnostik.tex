\section*{Neurodiagnostik}

\paragraph{Frage 1 (single choice)}

Für die Beurteilung der Wirkung ionisierender Strahlung auf den menschlichen Körper ist  die Äquivalentdosis eine wichtige Größe. In welcher Einheit wird die Äquivalentdosis angegeben?
\begin{benumerate}
  \item Bq (Becquerel)
  \item C/kg (Coulomb pro Kilogramm)
  \bolditem Sv (Sievert)
  \item Gy/Rd (Gray pro Rad)
\end{benumerate}

\paragraph{Frage 2 (single choice)}

In einer Röntgenanlage zur medizinischen Diagnostik mit einer typischen Röntgenröhre wird durch einen Generator aus der Netzspannung 220 Volt eine Hochspannung (in der Größenordnung von 100 kV) erzeugt. Für welchen der folgenden Vorgänge wird dabei aus prinzipiellen Gründen umbedingt Hochspannung benötigt?

\begin{benumerate}
  \item Abschirmung der Röntgenröhre
  \bolditem Beschleunigung der freien Elektronen in der Röntgenröhre
  \item Einstellung der Blenden
  \item hochtourige Drehung des Anodentellers in der Röntgenröhre
  \item Kühlung der Röntgenröhre
\end{benumerate}

\paragraph{Frage 3 (single choice)}

Welche Aussage zur Elektroenzephalographie (EEG) trifft zu?

\begin{benumerate}
  \item Es entspricht im Wesentlichen summierten synchronen Aktionspotentialen von oberflächlichen alpha-Motoneuronen.
  \item Bei Erwachsenen im inaktiven Wachzustand mit geschlossenen Augen lässt sich keinerlei elektrische Aktivität nachweisen (Nulllinie).
  \item Bei wachen, aufmerksamen Erwachsenen mit offenen Augen finden sich überwiegend Subdelta-Wellen.
  \bolditem Bei einem epileptischen Anfall können typische spikes/sharp waves auftreten.
  \item Aufgrund geringer Synchronisierung der neuronalen Aktivität ist die dominierende Frequenz der Wellen im Wachzustand bei Kleinkindern um ein vielfaches höher als bei Erwachsenen.
\end{benumerate}

\paragraph{Frage 4 (single choice)}

Welche Aussage zur Untersuchung des Gehirns und Rückenmarks mittels Magnetresonanztomographie (MRT) ist zutreffend?
\begin{benumerate}
  \item MRT ist eine typische nuklearmedizinische Untersuchung.
  \item MRT wird speziell bei Patienten mit Herzschrittmachern eingesetzt.
  \item Die unterschiedlichen Relaxationszeiten sind für den Bildkontrast egal.
  \item Die angewandte ionisierende Strahlung ist aufgrund der Abschirmung der Spulen unschädlich für den untersuchten Patienten.
  \bolditem Der Spin der Wasserstoffatome erzeugt durch ihren eigenen Drehimpuls ein kleines Magnetfeld.
\end{benumerate}

\paragraph{Frage 5 (single choice)}

Es soll die Nervenleitgeschwindigkeit ermittelt werden. Daher wird bei einem Patienten der \emph{N. ulnaris} am Oberarm und am Handgelenk gereizt und jeweils das Summenaktionspotential abgeleitet. Bei Reizung am Oberarm beginnt das Summenaktionspotential nach \( 10.5 \) ms und bei Reizung am Handgelenk nach \( 2.1 \) ms. Die beiden Reizorte sind exakt \( 42 \) cm voneinander entfernt. Die motorische Nervenleitgeschwindigkeit ist die mittlere Erregungsleitgeschwindigkeit zwischen den beiden Reizorten. Wie groß ist sie bei dem Patienten?

1. 20 m/s \enskip 2. 30 m/s \enskip 3. 40 m/s \enskip \textbf{4.} 50 m/s \enskip 6. 60 m/s

\paragraph{Frage 6 (single choice)}

Welche Aussage zur natürlich vorkommender Strahlung ist zutreffend?
\begin{benumerate}
  \item Terrestrische Strahlung ist auf der Erdoberfläche eine konstante Messgröße.
  \item Kosmische Strahlung entsteht in der inneren Atmosphäre und ist bei Langstreckenflügen geringer als am Boden.
  \item Die natürliche Strahlenbelastung ist eine Summation aus der inneren Strahlung und medizinischer Strahlenexposition und steigt in Deutschland kontinuierlich.
  \bolditem Die aufgenommene Menge an Radon und deren radioaktiven Zerfallsprodukten in der Atemluft, im Trinkwasser und in Nahrungsmitteln, verursachen den Hauptanteil der natürlichen Strahlungsexposition in Deutschland.
  \item Radionuklide wie Thorium und Polonium sind die beiden radioaktiven Stoffe die den größten Anteil der natürlichen Strahlenexposition in Europa verursachen.
\end{benumerate}

\paragraph{Frage 7 (one combination)}

Welche Verfahren sind in der medizinischen Diagnostik gebräuchliche Röntgenverfahren, die mit ionisierenden Strahlen arbeiten?

\begin{minipage}{.65\linewidth}
  \begin{benumerate}
    \item Coloskopie
    \item Mammographie
    \item Digitale Subtraktionsangiographie (DSA)
    \item Transcranielle Dopplersonographie (TCD)
    \item Intraoperative Durchleuchtung der Lendenwirbelsäule mittels Bildwandler
  \end{benumerate}
\end{minipage}%
\begin{minipage}{.3\linewidth}
  \begin{checklist}[leftmargin=7mm]
    \item 1, 3 richtig
    \item 2, 4 richtig
    \item[\checkedbox] 2, 3, 5 ist richtig
    \item nur 5 richtig
    \item alle richtig
  \end{checklist}
\end{minipage}

\paragraph{Frage 8 (single choice)}

Welche Aussage zur Dopplersonographie ist \textbf{falsch}? Bitte die falsche Aussage ankreuzen.
\begin{benumerate}
  \item Die Veränderung der Signallaufzeit zeigt sich durch eine Stauchung bzw. Dehnung der Frequenz des Schallsignals
  \item Das Dopplersignal liegt selbst bei niederen Flussgeschwindigkeiten von 8mm/s eindeutig im hörbaren Bereich.
  \bolditem Die Dopplerfrequenz ist das reflektierte Signal welches um eine konstante Frequenz (\( \Delta f \)) im Vergleich zur von der Sonde ausgesandten Geschwindigkeit verschoben ist
  \item Strömt das Blut auf die Schallsonde zu, verkürzt sich die Wellenlänge und die Schallfrequenz nimmt zu
  \item Man spricht vom Dopplereffekt, wenn Sender und Empfänger einer Welle sich relativ zueinander bewegen und aus ihrem Vorzeichen lässt sich daher die Flussrichtung rekonstruieren
\end{benumerate}

\paragraph{Frage 9 (single choice)}

Die Anwendung der Magnetresonanztomographie (MRT) in der bildgebenden Diagnostik von Gehirn und Rückenmark gehört heute zu den Standarduntersuchungsverfahren. Welche Aussage trifft \textbf{nicht} zu? Bitte die falsche Aussage ankreuzen.
\begin{benumerate}
  \item Die Auflösung ist bei klinischen Standardsystemen viel höher als bei der Positronen-Emissionstomographie (PET)
  \item Chemical-Shift-Artefakte entstehen durch unterschiedliche Präzessionsfrequenzen der Fett- und Wasserprotonen
  \item Verzerrungsartefakte durch lokale Magnetfeldinhomogenitäten können die Präzision der Navigationsrohdaten massiv vermindern
  \bolditem Nichtferromagnetische Titanimplantate können durch Erwärmung während der Untersuchung Gewebeschäden verursachen, daher ist eine MRT-Untersuchung bei Patienten mit Titan-Implantaten kontraindiziert
  \item Eine physikalische Grundlage für den Bildkontrast ist der unterschiedliche Gehalt an Wasserstoffatomen, woraus die spezifischen Relaxationszeiten verschiedener Gewebearten resultieren
\end{benumerate}

\paragraph{Frage 10 (single choice)}

Welche Aussage zur Strahlenbelastung des menschlichen Körpers ist korrekt?
\begin{benumerate}
  \item Je weicher die Röntgenstrahlung, desto geringer ist der Anteil der Strahlung die vom Gewebe absorbiert wird..
  \item Dichtes Gewebe des Körpers, zum Beispiel Knochen, absorbieren die Röntgenstrahlen besonders wenig.
  \item Röntgenstrahlen haben Wellenlängen von 10nm bis 100pm, je größer die Wellenlänge, desto größer die Frequenz und damit die Energie eines Strahlungsteilchens.
  \item Von der Anode der Röntgenröhre werden Protonen emittiert, durch die angelegte Hochspannung (20-600kV) beschleunigt, dringen in das Kathodenmaterial ein und erzeugen durch das Abbremsen die charakteristische Röntgenstrahlung.
  \bolditem Weiche Röntgenstrahlung mit einer typischen Röhrenspannung zwischen 25-35kV macht feinste Gewebeunterschiede sichtbar.
\end{benumerate}

\paragraph{Frage 11 (one combination)}

Welche Verfahren zählen in der Neurologie/Neurochirurgie zur elektrophysiologischen Diagnostik?

\begin{minipage}{.7\linewidth}
  \begin{benumerate}
    \item Nervenleitgeschwindigkeitsmessung (NLG)
    \item Elektroenzephalographie (EEG)
    \item Visuellvozierte Potentiale (VEP)
    \item Elektrokardiographie (EKG)
    \item Positron-Emissions-Tomographie (PET)
  \end{benumerate}
\end{minipage}%
\begin{minipage}{.25\linewidth}
  \begin{checklist}[leftmargin=7mm]
    \item 1, 3 richtig
    \item 2, 4 richtig
    \item[\checkedbox] 1, 2, 3 richtig
    \item 1, 2, 4 richtig
    \item alle richtig
  \end{checklist}
\end{minipage}

\paragraph{Frage 12 (one combination)}

Welche Aussage zu den Evozierten Potentialen (EP) ist korrekt?

\begin{minipage}{.7\linewidth}
  \begin{benumerate}
    \item Zur Darstellung evozierter Potentiale werden mehrere Realisierungen eines Potentials gemittelt.
    \item Evozierte Potentiale haben wesentlich größere Amplituden (500-1000 \( \mu \)V) als spontan ablaufende EEG-Signale (1-15 \( \mu \)V)
    \item Jeder Sinnesreiz löst in den sensorischen Arealen der Großhirnrinde elektrische Potentialänderungen aus.
    \item Mittels somatosensorisch evozierten Potentialen (SSEP) wird die Leitfähigkeit von motorischen peripheren Nerven und die jeweilige Muskelfunktion geprüft.
    \item Akustisch evozierte Potentiale (AEP) prüfen die Leitfähigkeit des \emph{nervus vestibulocochlearis} (Hörnerv) und olfaktorisch evozierte Potentiale (OEP) dienen der objektiven Prüfung des Geruchssinns (\emph{nervus olfactorius}).
  \end{benumerate}
\end{minipage}%
\begin{minipage}{.25\linewidth}
  \begin{checklist}[leftmargin=7mm]
    \item 1, 3 richtig
    \item 1, 2, 3 richtig
    \item[\checkedbox] 1, 3, 5 richtig
    \item 1, 3, 4 richtig
    \item alle richtig
  \end{checklist}
\end{minipage}

\paragraph{Frage 13 (one combination)}

Welche Verfahren sind in der medizinischen Diagnostik gebräuchliche Ultraschallverfahren?

\begin{minipage}{.7\linewidth}
  \begin{benumerate}
    \item Transkranielle Dopplersonographie
    \item Echokardiographie
    \item Nierensonographie
    \item Diaphanoskopie
    \item Mammographie
  \end{benumerate}
\end{minipage}%
\begin{minipage}{.25\linewidth}
  \begin{checklist}[leftmargin=7mm]
    \item 1, 3 richtig
    \item 1, 2, 4 richtig
    \item[\checkedbox] 1, 2, 3 richtig
    \item 1, 3, 5 richtig
    \item alle richtig
  \end{checklist}
\end{minipage}