\section*{Motorik und Sensorik}

\paragraph{Frage 1 (one combination)}

Die Informationsweitergabe im Zentralen Nervensystem (ZNS) basiert auf einer elektrochemischen Impulsweitergabe, hierbei werden unterschiedliche Botenstoffe (sogenannte Neurotransmitter) in den synaptischen Spalt abgegeben. Welches sind typische Neurotransmitter?

\begin{minipage}{.5\linewidth}
  \begin{benumerate}
    \item Glycin
    \item Adenosin
    \item Noradrenalin
    \item Dopamin
    \item Serotonin
  \end{benumerate}
\end{minipage}
\hfill
\begin{minipage}{.5\linewidth}
  \begin{checklist}
    \item 1, 3 und 5 ist richtig
    \item 2 und 4 ist richtig
    \item 1, 3, 4 und 5 ist richtig
    \item nur 5 ist richtig
    \item[\checkedbox] alle Aussagen sind richtig
  \end{checklist}
\end{minipage}

\paragraph{Frage 2 (single choice)}

Welche Aussage zum zellulären Aufbau eines Nervens ist \textbf{falsch}? Bitte kreuze die falsche Aussage an.
\begin{benumerate}
  \item Die Nervenenden werden als Synapsen bezeichnet und dienen als mobile Übertragungsstationen
  \item Die schneller leitenden Nervenfasern sind typischerweise von Myelin umgeben
  \item Die Nervenzelle besteht in der Regel aus drei Anteilen: dem Zellkörper, den Zellfortsätzen (Dendriten) und dem Axon
  \bolditem Die Mitochondrien in der Nervenzelle dienen als Speicher, insbesondere in den motorischen Endplatten wird hier Ca++ gespeichert
  \item Es existieren einige spezialisierte Nervenzellen, die keine Dendriten besitzen, wie zum Beispiel die Stäbchen und Zapfen in der Netzhaut
\end{benumerate}

\paragraph{Frage 3 (single choice)}

Welcher der folgenden Vorgänge wird durch eine schnelle passive Dehnung eines Skelettmuskels ausgelöst?
\begin{benumerate}
  \item Abnahme der Aktionspotentialfrequenz in la-Fasern desselben Muskels
  \item Dehnung der Muskelspindeln am antagonistischen Muskel
  \bolditem Acetylcholin-Freisetzung an motorischen Endplatten desselben Muskels
  \item Kontraktion des antagonistischen Muskels
\end{benumerate}