\section*{Neurologische Krankheitsbilder}

\paragraph{Frage 1 (single choice)}

Wenn das Rückenmark in Höhe der unteren Brustwirbelsäule (thorakales Rückenmarkssegment) komplett durchtrennt ist (Querschnittsyndrom), so zählt zu den bleibenden neurologischen Störungen typischerweise

\begin{benumerate}
  \item Akute Erblindung
  \bolditem Unfähigkeit zur willkürlichen Harnblasenentleerung
  \item aufgehobenes Temperaturempfinden am linken Daumen
  \item Sprachstörung durch Lähmung des \emph{musculus temporalis}
  \item Lähmung der Augenmuskulatur
\end{benumerate}

\paragraph{Frage 2 (single choice)}

Welche Aussage zu epileptischen Anfällen ist \textbf{nicht} korrekt? Bitte die falsche Antwort ankreuzen.
\begin{benumerate}
  \item Krampfanfälle können durch Schlafentzug und rhythmische Lichtimpulse ausgelöst werden, gelegentlich kann eine aura vor dem epileptischen Anfall auftreten.
  \item In der Elektroenzephalographie (EEG) können auch im freien Intervall typische spike- und wave-Muster (charakteristische Wellen) zu sehen sein.
  \item Bei einem \emph{grand mal}-Anfall kann es durch Stürzt zu schweren Schädelhirntraumen kommen.
  \bolditem Fokale Anfälle sind gekennzeichnet durch asynchrone Entladungen von Neuronengruppen im gesamten oberflächlichen Cortex.
  \item Bei typischen Absencen korreliert die einige Sekunden dauernde ``Abwesenheit'' in der Regel mit dem Auftreten von 3/s spike-wave-Komplexen in der Elektroenzephalographie (EEG).
\end{benumerate}

\paragraph{Frage 3 (Freitext)}

Beschreiben Sie die Problematik der Einschätzung von ``gutartigen'' beziehungsweise ``bösartigen'' Tumoren im Gehirn. Nennen Sie stichpunktartig objektive Kriterien für diese Klassifikationen und subjektive Komponenten did diese Einschätzung bei Tumoren im zentralen Nervensystem schwierig machen.