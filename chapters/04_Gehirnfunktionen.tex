\section*{Gehirnfunktionen}

\paragraph{Frage 1 (single choice)}

Welche der Aussagen zum Gedächtnis trifft am wahrscheinlichsten zu?

\begin{benumerate}
  \item Das Kurzzeitgedächtnis kann deutlich mehr Informationen speichern als das Langzeitgedächtnis.
  \item Das prozedurale Gedächtnis (Verhaltensgedächtnis) ist hauptsächlich im Kleinhirn.
  \item Das Kurzzeitgedächtnis funktioniert rein elektrisch.
  \item Das Ultrakurzzeitgedächtnis kann nur Informationen speichern, die durch das enterorhinale System vorverarbeitet wurden.
  \bolditem Bei einer Schädigung beider Hippocampi und des Fornix kommt es zur Störung des Gedächtnisses für neue Informationen.
\end{benumerate}

\paragraph{Frage 2 (single choice)}

Welche der nachstehenden Aussagen bezieht sich auf diejenige Funktion, die für das Kleinhirn (\emph{cerebellum}) am ehesten kennzeichnend ist?
\begin{benumerate}
  \item Das Cerebellum ist für die Sprachproduktion zuständig.
  \bolditem Das Cerebellum ist an der sensomotorischen Koordination beteiligt.
  \item Das Cerebellum ist Hauptspeicherort für Geruchs- und Geschmacksassoziationen.
  \item Im Cerebellum werden die Farben aus visuellen Informationen verarbeitet.
  \item Im Cerebellum werden negative Gefühle wie Angst verarbeitet und gespeichert.
\end{benumerate}

\paragraph{Frage 3 (one combination)}

Der Morbus Alzheimer ist eine der häufigsten Demenzformen. Welche Aussagen zur Erkrankung M. Alzheimer sind zutreffend?

\begin{minipage}{.7\linewidth}
  \begin{benumerate}
    \item Einer der Pathomechanismen der Erkrankung ist das Ungleichgewicht der Neurotransmitter, wobei die Konzentration von Glutamat erhöht und von Acetylcholin verringert ist.
    \item Aufgrund der Ablagerungen von beta-Amyloid im Gehirn nimmt das Hirnvolumen von Alzheimerpatienten stetig zu.
    \item Bei erkrankten Patienten zeigt sich eine zunehmende Störung der kognitiven Funktionen.
    \item Aufgrund der Störung in den Mitochondrien nimmt der Hirnstoffwechsel im Spätstadium der Erkrankung zu, was sowohl in der Positronenemissionstomographie (PET) als auch in der funktionellen Kernspintomographie (fMRT) eindeutig zu erkennen ist.
    \item Bei histologischen Untersuchungen des Gehirns von Alzheimerpatienten finden sich vermehrt sogenannte senile Plaques.
  \end{benumerate}
\end{minipage}%
\begin{minipage}{.3\linewidth}
  \begin{checklist}[leftmargin=7mm]
    \item 1, 3 richtig
    \item 2, 4 richtig
    \item[\checkedbox] 1, 3, 5 richtig
    \item nur 5 richtig
    \item alle richtig
  \end{checklist}
\end{minipage}